\chapter{Zdrojové kódy pre simulácie} \label{ch:priloha_sim}

\section{\textsc{Spice} - príklad netlistu}
\myfig{schema_input}{Príklad všeobecnej simulačnej schémy pre generovanie SPICE netlistu (vytvorená v programe Xcircuit).}{\label{fig:append_schema_input}}

\subsection{input.spc}

\begin{lstlisting}
*SPICE circuit <input> from XCircuit v3.8 rev 78

.include gce.sp
.include control.sp
C5 int4 int10 000p
C2 int7 ne 000p
L3 int10 nc 00n
Rd int11 int4 r='v(int11 ) > v(int4 ) ? .1u : 10MEG'
V1 int4 int5 303
L1 ne nepar 00n
Va1 nepar 0 0
R1 0 int5 0
Rce1 int7 ne r='1/v(ngce)'
L4 int11 int10 00n
L2 nc int7 000n
I1 int4 int10 10.3
C1 int7 0 000p
C3 int10 0 000p
C4 int10 int7 000p
C6 int4 int11 000p

.end
\end{lstlisting}

\newpage
\subsection{gce.sp}
(generované pomocou \textit{Octave})

\begin{lstlisting}
Btime ntime 0 v= 'time'

**************************************
*** vypocet gce podla mojej krivky ***
**************************************
Bgce ngce 0 v=
+	v(ntime) <0.0001948?
+		-3423555918453.736*0.0001948*0.0001948+ 1334502097.012*0.0001948+-130038.128
+	:v(ntime) <0.000195?
+		-3423555918453.736*v(ntime)*v(ntime)+ 1334502097.012*v(ntime)+-130038.128
+	:v(ntime) <0.00019515?
+		-270156826283465.4*v(ntime)*v(ntime)+105360477539.367*v(ntime)+-10272570.734
+	:v(ntime) <0.00019521?
+		602514803859639.9*v(ntime)*v(ntime)+-235243259705.487*v(ntime)+22961838.928
+	:v(ntime) <0.00019524?
+		156172912727445.8*v(ntime)*v(ntime)+-60982458569.656*v(ntime)+5953113.433
+	:v(ntime) <0.0001954?
+		155024509820.893*v(ntime)*v(ntime)+-60593578.438*v(ntime)+5920.972
+	:v(ntime) <0.0001957?
+		16666666667.291*v(ntime)*v(ntime)+-6523333.334*v(ntime)+638.309
+	:
+		16666666667.291*0.0001957*0.0001957+-6523333.334*0.0001957+638.309
**************************************
\end{lstlisting}


\subsection{control.sp}

(generované pomocou \textit{Octave})

\begin{lstlisting}
.control
tran 1.9e-09 0.0001962 0.0001943
set nobreak
print ngce nc i(Va1) > /tmp/patocka__/data.data
.endc
\end{lstlisting}


\newpage

\section{Knižnica \textit{Octave} (\textsc{Matlab}) funkcií pre generovanie modelu $g_{CE}$}

Vstupnými (zmeranými) datami je tabuľka vo forme (\ref{eq:tab_predpis_t012}), syntaxou Matlabu zapísaná do štruktúry:
\begin{verbatim}
tab_on.tus = [t0, t1, t2, t3];
tab_on.G = [G0, I1/U1, I2/U2, I3/U3];
tab_on.t = tab_on.tus * 1e-6;
\end{verbatim}
Z tabuľky sú funkciou \texttt{tab2abc} dopočítané konštanty vystupujúce vo vzťahu (\ref{eq:krivka_subintervaly_gong1g2goff}) a následne sú pomocou funkcie \texttt{write\_spice\_model} zapísané časovo závislé SPICE výrazy na jednotlivých intervaloch daných tabuľkou zapísané do súbora na disku, ktorý je neskôr zahrnutý do simulačného netlistu. Z prostredia \textit{Octave} alebo \textsc{Matlab} je tiež možné pomocou príkazu \texttt{system(command)} spustiť simulátor (príkazom pre operačný systém) a následne čítať a ďalej spracúvať výsledky simulácie.

\subsection*{compute\_and\_write\_spice\_model\_my.m}
\begin{lstlisting}
function ret = compute_and_write_spice_model_my(tab_t, tab_G, str_outfilename)
	ret.abc = abc = tab2abc(tab_t, tab_G);
	ret.writestatus = write_spice_model_my(tab_t, abc, str_outfilename);
endfunction
\end{lstlisting}

\subsection*{write\_spice\_model\_my.m}
\begin{lstlisting}
function ret = write_spice_model_my(tab_t, abc, str_filename)

	if((FID = fopen(str_filename, "w")) == -1)
		puts("nepodarilo sa otvorit subor: ");
		puts(str_filename);
		puts("\n");
		ret = -1;
	else
		ret =1;
	endif
	
	for i=1:numel(tab_t)
		eval(strcat("str_t", num2str(i-1), " = num2str(tab_t(", num2str(i), "));"));
	endfor
	
	for i=1:numel(abc)/3
		eval(strcat("str_a", num2str(i), " = num2str(abc(", num2str((i-1)*3+1), "));"));
		eval(strcat("str_b", num2str(i), " = num2str(abc(", num2str((i-1)*3+2), "));"));
		eval(strcat("str_c", num2str(i), " = num2str(abc(", num2str((i-1)*3+3), "));"));
	endfor

	
	fputs(FID, "Btime ntime 0 v= 'time'\n\n"); 
	
	fputs(FID, "**************************************\n*** vypocet gce podla mojej krivky ***\n**************************************\n");
	fputs(FID, "Bgce ngce 0 v=\n");
	fputs(FID, strcat("+\tv(ntime) < ", str_t0, "?\n"));
	fputs(FID, strcat("+\t\t", str_a1, "*", str_t0, "*", str_t0, "+", str_b1, "*", str_t0, "+", str_c1, "\n"));
	
	for i=1:numel(abc)/3
		str_ti = num2str(tab_t(i+1));
		str_ai = num2str(abc((i-1)*3+1));
		str_bi = num2str(abc((i-1)*3+2));
		str_ci = num2str(abc((i-1)*3+3));

		feval("fputs", FID, strcat("+\t:v(ntime) < ", str_ti, "?\n"));
		feval("fputs", FID, strcat("+\t\t", str_ai, "*v(ntime)*v(ntime)+", str_bi, "*v(ntime)+", str_ci, "\n"));
	endfor


	fputs(FID, "+\t:\n");


	feval("fputs", FID, strcat("+\t\t", str_ai, "*", str_ti, "*", str_ti, "+", str_bi, "*", str_ti, "+", str_ci, "\n"));
%	fputs(FID, strcat("+\t\t", str_a2, "*", str_t2, "*", str_t2, "+", str_b2, "*", str_t2, "+", str_c2, "\n"));
	fputs(FID, "**************************************\n");
	
	if(fclose(FID) == -1)
		puts("nepodarilo sa zatvorit subor: ");
		puts(str_filename);
		puts("\n");
		ret = -2;
	else
		ret = 2;
	endif
endfunction
\end{lstlisting}

\subsection*{tab2abc\_off.m}
\begin{lstlisting}
function abc = tab2abc_off(tab_t, tab_G)
	switch numel(tab_t)
		case(3)
			abc = tab2abc_off_2(tab_t, tab_G);
		case(4)
			abc = tab2abc_off_3(tab_t, tab_G);
		case(5)
			abc = tab2abc_off_4(tab_t, tab_G);
		case(6)
			abc = tab2abc_off_5(tab_t, tab_G);
		case(7)
			abc = tab2abc_off_6(tab_t, tab_G);
		otherwise
			error("malo alebo vela bodov v tabulke. mozne je od 3 do 7");
	endswitch
endfunction
\end{lstlisting}

\subsection*{tab2abc\_off\_2.m}
\begin{lstlisting}
function abc12= tab2abc_off_2(tab_t012, tab_G012)

	t0=tab_t012(1);
	t1=tab_t012(2);
	t2=tab_t012(3);
	G0=tab_G012(1);
	G1=tab_G012(2);
	G2=tab_G012(3);

	V = [G0; G1; G1; G2; 0; 0];

	M = [	t0^2,	t0,	1,	0,	0,	0;
		t1^2,	t1,	1,	0,	0,	0;
		0,	0,	0,	t1^2,	t1,	1;
		0,	0,	0,	t2^2,	t2,	1;
		2*t1,	1,	0,	-2*t1,	-1,	0;
		0,	0,	0,	2*t2,	1,	0];
	abc12 = X = M \ V;
endfunction
\end{lstlisting}

\subsection*{tab2abc\_on.m}
\begin{lstlisting}
function abc = tab2abc_on(tab_t, tab_G)
	switch numel(tab_t)
		case(3)
			abc = tab2abc_on_2(tab_t, tab_G);
		case(4)
			abc = tab2abc_on_3(tab_t, tab_G);
		case(5)
			abc = tab2abc_on_4(tab_t, tab_G);
		case(6)
			abc = tab2abc_on_5(tab_t, tab_G);
		case(7)
			abc = tab2abc_on_6(tab_t, tab_G);
		otherwise
			error("malo alebo vela bodov v tabulke. mozne je od 3 do 7");
	endswitch
endfunction
\end{lstlisting}

\subsection*{tab2abc\_on\_2.m}
\begin{lstlisting}
function abc12= tab2abc_on_2(tab_t012, tab_G012)

	t0=tab_t012(1);
	t1=tab_t012(2);
	t2=tab_t012(3);
	G0=tab_G012(1);
	G1=tab_G012(2);
	G2=tab_G012(3);

	V = [G0; G1; G1; G2; 0; 0];

	M = [	t0^2,	t0,	1,	0,	0,	0;
		t1^2,	t1,	1,	0,	0,	0;
		0,	0,	0,	t1^2,	t1,	1;
		0,	0,	0,	t2^2,	t2,	1;
		2*t1,	1,	0,	-2*t1,	-1,	0;
		2*t0,	1,	0,	0,	0,	0];
	abc12 = X = M \ V;
endfunction
\end{lstlisting}


