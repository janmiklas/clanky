\chapter*{Úvod} \label{ch:uvod} \addcontentsline{toc}{chapter}{Úvod}
\markboth{\MakeUppercase{Úvod}}{\MakeUppercase{Úvod}} % aby to aj v hlavicke strany pisalo uvod a nie nieco predchadzajuce, napr. zoznam obrazkov


\lettrine{S}{pínacie} straty tvoria podstatnú časť celkových strát v spínacích polovodičových prvkoch pracujúcich pri vysokých frekvenciách. Ich analýza je preto prirodzene potrebná z hľadiska aplikačného ako aj pri vývoji súčiastok.

Analytické dynamické modely tranzistorov používané v bežných obvodových simuláciách, či už na základe \uv{charge control} prístupu (BJT, \cite{gummel-poon}, \cite{pierret}) alebo iné nie je možné použiť pre výkonové spínacie tranzistory. Behom spínania totiž prechádza tranzistor viacerými odlišnými režimami, ktorých hranice navyše nie sú ostro určené (podrobnejšie popisované napr. v \cite{baliga}), nehovoriac o skutočnej priestorovej komplexnosti polovodičových štruktúr oproti základným analytickým predstavám.

Pre jednotlivé typy spínacích tranzistorov existujú pomerne dôveryhodné ekvivalentné modely, ako napr. známy Hefnerov model \cite{hefner} IGBT tranzistora, ich použitie resp. zostavenie je však nie celkom priamočiare, čo môže byť zvlášť pre aplikačných inžinierov, ktorých zameraním nie sú obvodové simulácie či charakterizácia súčiastok, faktorom rozhodujúcim o samotnom použití alebo nepoužití simulátora.

Vedúci tejto práce so spolupracovníkmi \cite{valsa-patocka-petru} navrhli v časoch bipolárnych tranzistorov jednoduchú analýzu zmeraných spínacích priebehov pomocou predstavy časovo premennej vodivosti tranzistora $g_{CE}$. Táto predstava je natoľko základná, že nie je obmedzená na jeden konkrétny typ súčiastky, ale dá sa použiť (snáď s istými upresneniami) pre ľubovoľnú spínaciu súčiastku, teda aj moderné rýchle unipolárne tranzistory.

Konkrétnym spracovaním zmeraných priebehov a priebehu vodivosti $g_{CE}$ sa venuje podstatná časť tejto práce.

Samotné hodnoverné meranie prepínacích strát sa ostáva pri trende čoraz extrémnejších parametrov (rýchlosť, veľkosť prúdu) stále technickou výzvou.
Cieľom diplomovej práce je zostavenie meracieho pracoviska, hodnoverné zmeranie spínacích priebehov vybraných súčiastok a následné vytvorenie jednoduchého, ale široko platného simulačného modelu spínacích dejov konkrétnych súčiastok.
