\chapter{Časovo premenná vodivosť $g_{CE}$} \label{ch:gce}

\lettrine{A}{ko} bolo naznačené v úvode, s odkazom na článok \cite{valsa-patocka-petru}, spínací tranzistor možno jednoduchým spôsobom fenomenologicky \uv{dvojpólovo} modelovať ako časovo premennú vodivosť medzi kolektorom a emitorom. Myšlienka neskokovej zmeny tranzistora z nevodivého stavu do vodivého a naopak je pomerne prirodzená, jej dôsledné rozvinutie je však mimoriadne užitočné, a to nielen pre jednoduchosť modelu, ale aj pre prehľadnú analýzu celého tranzistorového spínača so záťažou (Obr. \ref{fig:schema_zakladna_ch_gce}) za predpokladu, že časový priebeh vodivosti $g_{CE}(t)$ nezávisí na kolektorovom obvode. 

\myfigtex{obr/schema_zakladna}{Tranzistorový spínač}{\label{fig:schema_zakladna_ch_gce}}
\myfigtex{obr/plots/simulacia_zakladna_300V100V}{Predstava vodivosti $g_{CE}(t)$ behom vypínacieho deja.}{\label{fig:simulacia_zakladna_300V100V}}

Demonštrovať to možno na príklade analýzy spínania indukčnej záťaže, zabezpečujúcej po krátku dobu zapnutia / vypnutia konštantný prúd.  Z modelu je evidentné, že zlom v priebehoch napätia aj prúdu (na Obr. \ref{fig:simulacia_zakladna_300V100V} je zobrazený príklad vypínacieho deja) nie je priamo daný tranzistorom, ale súčinnosťou celého obvodu, teda prakticky nulovou diodou, napäťovým a prúdovým zdrojom (medziobvod a tlmivka). Je samozrejmé, že časová poloha tohto zlomu (resp. rýchlosť deja) je určená vlastnosťami tranzistora, nijako však nesúvisí s nejakým \uv{zlomom} v dejoch vo vnútri tranzistora. Dochádza k nemu v momente, kedy je nulová dioda prepólovaná do priepustného smeru (napätie na anóde presiahne napätie na katóde). Do tohto momentu bol konštantný prúd záťaže $i_L$ nútený tiecť klesajúcou vodivosťou $g_{CE}$ spôsobujúcou zväčšujúci sa úbytok $u_{CE}$. Od okamihu prepólovania diody je na kolektore udržované konštantné napätie medziobvodu, nezávisle na veľkosti prúdu $i_D(t)$. Vodivosť $g_{CE}$ naďalej klesá, čím postupne núti čoraz väčší podiel konštantného prúdu záťaže odtekať cestou cez diodu, až po ustálený stav, kedy je tranzistor zavretý.


Ďalej z tohto jednoduchého modelu plynie, že doby známe ako $t_d$ (\textit{delay time}) a $t_f$ (\textit{fall time}), ktoré sa menia v závislosti na veľkosti spínaného napätia, sú spolu zviazané vzťahom $t_d + t_f = t_{off}$, kde $t_{off}$ je celková vypínacia doba, ktorá je nezávislá na napätí (Obr. \ref{fig:simulacia_zakladna_300V100V}). Závislá je na spôsobe budenia tranzistora, teplote či veľkosti spínaného prúdu (ako bude pozorované aj v meraniach prezentovaných v kapitole \ref{ch:vysledky}), ktoré ovplyvňujú tvar krivky časového priebehu $g_{CE}(t)$.

Autori článku \cite{valsa-patocka-petru} v čase jeho vzniku pozorovali pri množstve meraní bipolárneho tranzistora za rozličných podmienok tvar krivky $g_{CE}(t) = \frac{i_C(t)}{u_{CE}(t)}$ ako hladký, tvarovo typicky vystihnutý na analyzovanom Obr. \ref{fig:simulacia_zakladna_300V100V}. Nakoľko je jej priebeh hladký\footnote{Každopádne platí predchádzajúca analýza a možno tvrdiť minimálne toľko, že v okamihu \uv{zlomu} v priebehoch je priebeh krivky hladký}, najmä pri moderných rýchlych unipolárnych súčiastkach, je presne analyticky vyjadrovať nemožné; podstatná je myšlienka, že vypínací i zapínací dej možno analyzovať ako \uv{číru} vodivosť vyjadriteľnú Ohmovým zákonom ako $\frac{i(t)}{u(t)}$ a prehnutý tvar priebehov nie je teda daný nijakými zotrvačnými javmi, ako napr. kapacitou.

Nakoniec, v prípade, že by to neplatilo, neplatila by ani všeobecne zaužívaná \uv{definícia} stratového výkonu počas spínania v podobe $p(t)~=~u_{CE}(t)~\cdot~i_C(t)$.

