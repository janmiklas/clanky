\chapter*{Záver} \label{ch_zaver} \addcontentsline{toc}{chapter}{Záver}
\markboth{\MakeUppercase{Záver}}{\MakeUppercase{Záver}} % aby to aj v hlavicke strany pisalo zaver a nie nieco predchádzajuce, napr. zoznam obrazkov


\lettrine{V}{ rámci} práce sa podarilo vytvoriť meracie pracovisko na presné oscilografické zaznamenanie spínacích dejov; otestované bolo na meraniach bipolárnych a IGBT tranzistorov. Analýza nameraných priebehov naznačuje, že prítomné parazitné vplyvy nie sú významným spôsobom zapríčinené nesprávnym spôsobom snímania.

Ako snímač prúdu je po viacerých pokusoch (s prúdovým transformátorom, Rogowského cievkou aj drahými komerčnými snímačmi) osvedčený SMD bočník s pomerne veľkou hodnotou odporu ($1\un{\Omega}$). Problémy so šírkou snímaného pásma pri proporcionálnej súčiastke akou je odpor nie sú a derivačný charakter spôsobený parazitnou indukčnosťou je potlačený práve veľkou hodnotou odporu. Príliš veľkú hodnotu však voliť nemožno, a to nie len kvôli úbytku napätia, ale aj pre tlmenie, ktorým by mohol skresľovať kmitavé javy v meranom obvode.

Z meraní vyplynula potreba oddeleného silového emitorového kontaktu od riadiaceho. Dôvodom je indukčnosť prívodu a ňou indukované napätie pri veľkých zmenách prúdu (tj. pri spínaní), ktoré spôsobuje rozdiel medzi potenciálom riadiacej (meracej) zeme a skutočným potenciálom emitora na čipe. Dôsledkom toho je nie len skreslený meraný údaj, ale hlavne možné spomalenie spínacieho deja, ako bolo popísané v kapitole \ref{ch:vysledky}. Tento jav býva pozorovaný najmä pri veľkých mnohočipových výkonových moduloch \cite{khanna}. Pozorovaný bol však v tejto práci aj u diskrétnych jednočipových súčiastkach. Manipuláciou s geometrickým usporiadaním resp. dĺžkou prívodov bolo možné ovplyvňovať spínacie deje z pohľadu času rádovo aj o $30\%$. Aj pri úplnom skrátení vývodov z púzdra sú ale elektródy čipu kontaktované s vývodmi pomocou bondovacích drátov, ktorých indukčnosť (ako sa ukázalo) rozhodne nemožno zanedbať. To ilustruje nevhodnosť klasických trojvývodových súčiastok pre rýchle aplikácie.

Parazitné javy (ako bolo ilustrované v kapitole \ref{ch:parazity}) majú za následok väčšie či menšie skreslenie skutočného prúdu a napätia na čipe. To však znamená, že nie všetká energia $\int u_{CE}(t) i_C(t) \dif t$ počas jedného prechodného deja je skutočnou stratovou energiou. Vo všeobecnosti ale možno usúdiť, že množstvo energie, ktoré by sa akumulovalo v parazitnom prvku počas jedného z dejov (čím by sa stalo meranie pesimistickým), sa zase uvoľní a zmarí na teplo počas druhého z dejov, čím súčiastke naopak \uv{prihorší}. Výslednú stratovú bilanciu teda možno považovať viac-menej za neskreslenú aj na pri nedokonalých meraniach.
Navyše, správnym návrhom aplikácie alebo merania je možné najpodstatnejšie parazitné vplyvy odstrániť.

V neposledom rade je súčasťou práce vytvorený \uv{dvojpólový} simulačný model aproximujúci časovú zmenu vodivosti tranzistora $g_{CE}$. Svojou jednoduchosťou oprosťuje simulácie od potrebného veľkého výpočtového výkonu (a dlhých časov simulácií), ako aj od problémov s konvergenciou výpočtov. Zostavenie modelu je veľmi priamočiare; vychádza priamo zo zmeraných priebehov.

Idealizované priebehy produkované takýmto modelom je veľmi jednoducho možné korigovať pridaním parazitných prvkov. Na obrázkoch v stati \ref{sec:vysl_IGBT} zreteľne vidno, že pridané obvodové prvky korigujú idealizované priebehy presne podľa očakávania.

% zhodnotit, ze sa da simulovat, a ze pridanie parazit (kond.) koriguje priebehy podla ocakavania
