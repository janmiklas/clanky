\documentclass[conference]{IEEEtran}
\IEEEoverridecommandlockouts
% The preceding line is only needed to identify funding in the first footnote. If that is unneeded, please comment it out.
\usepackage{cite}
\usepackage{amsmath,amssymb,amsfonts}
\usepackage{algorithmic}
\usepackage{graphicx}
\usepackage{textcomp}
\usepackage{xcolor}
\def\BibTeX{{\rm B\kern-.05em{\sc i\kern-.025em b}\kern-.08em
    T\kern-.1667em\lower.7ex\hbox{E}\kern-.125emX}}

%%% MOJE
\graphicspath{{obr/}{obr/plots/}{spice/}}
\DeclareGraphicsExtensions{.ps, .eps}
%\usepackage[slovak]{babel}
\usepackage[utf8]{inputenc} 

\newcommand{\dif}{\, \mathrm{d}}	% diferencia (na derivacie)
\newcommand{\difp}{\partial}		% parc. diferencia 
\newcommand{\dxdt}[2]{\frac{\mathrm{d} #1}{\mathrm{d} #2}}
\newcommand{\dxdtp}[2]{\frac{\partial #1}{\partial #2}}
\newcommand{\un}[1]{\, \mathrm{#1}}	% jednotky velicin, v math mode
\newcommand{\E}[1]{\cdot 10^{#1}}
\newcommand{\degree}{^\circ}
\newcommand{\diameter}{\emptyset}
\newcommand{\cpx}{\widehat}		% komplexne fazory
\newcommand{\Ohm}{\Omega}

\newcommand{\myfig}[3]
{
    \begin{figure}[!ht]
    %\begin{figure}[ht]
    %\begin{figure}[htpb]
	\centering
	\includegraphics{#1}
	\caption{#2}
	%\label{fig:#3}
	#3
    \end{figure}
}
\newcommand{\myfigsc}[3]
{
    \begin{figure}[!ht]
	\centering
	\includegraphics[width=3.5in]{#1}
	\caption{#2}
	%\label{fig:#3}
	#3
    \end{figure}
}

\newcommand{\myfigtex}[3]
{
    \begin{figure}[!ht]
        \centering
        \input{#1}
        \caption{#2}
        %\label{fig:#3}
        #3
    \end{figure}
}


\begin{document}

\title{Abcd Efgh
%{\footnotesize \textsuperscript{*}Note: Sub-titles are not captured in Xplore and should not be used}
\thanks{}
}

\author{\IEEEauthorblockN{1\textsuperscript{st} Jan Miklas}
\IEEEauthorblockA{\textit{Department of Power Electrical and Electronic Engineering} \\
\textit{Faculty of Electrical Engineering and Communication}\\
\textit{Brno University of Technology}\\
Brno, Czech Republic \\
jan.miklas@vutbr.cz}
\and
\IEEEauthorblockN{2\textsuperscript{nd} Petr Prochazka}
\IEEEauthorblockA{\textit{Department of Power Electrical and Electronic Engineering} \\
\textit{Faculty of Electrical Engineering and Communication}\\
\textit{Brno University of Technology}\\
Brno, Czech Republic \\
prochazkap@feec.vutbr.cz}
}

\maketitle

\begin{abstract}
    Aaa.\\
    Bbb.\\
    Ccc.\\
\end{abstract}

\begin{IEEEkeywords}
\end{IEEEkeywords}


\section{Introduction}
\section{Methodology}
\section{Measurement and Interpretation of Test Results} \label{sec:results}
\myfigtex{obr/Tzap}{}{\label{fig:Tzap}}
\section{Conclusion}


%\section*{References}


\section*{Acknowledgment}
This research work has been carried out in the Xxxxxx
Xxxxxx Xxxxxx Xxxxxx Xxxxxx Xxxxxx Xxxxxx Xxxxxx
(XXXXXXXXXXXXX). Authors gratefully acknowledge fi-
nancial support from the Xxxxxx Xxxxxx Xxxxxx Xxxxxx
Xxxxxx Xxxxxx Xxxxxx Xxxxxx (XXXXXXXXXXXXX).


\begin{thebibliography}{00}
	\bibitem{chenming} C. Hu and M. J. Model, "A model of power transistor turn-off dynamics," 1980 IEEE Power Electronics Specialists Conference, Atlanta, Georgia, USA, 1980, pp. 91-96.
	\bibitem{baliga} B. J. Baliga, Fundamentals of Power Semiconductor Devices, New York: Springer. 2008.
	\bibitem{pierret} R. F. Pierret, {Semiconductor Device Fundamentals}, Addison-Wesley Publishing Comany, 1996, ISBN 0-201-54393-1
	\bibitem{gummel} H. K. Gummel, H. K., ``A Charge Control Relation for Bipolar Transistors'', Bell Syst. Tech. J., vol. 49, 1970
	\bibitem{sze} S. M. Sze and M. K. Lee, Semiconductor Devices: Physics and Technology, 3rd ed., New York: John Wiley \& Sons, Inc. 2012.
	%\bibitem{prochazka2} P. Prochazka, I. Pazdera, J. Miklas and R. Cipin, "Analysis of Power Transistor Switching Process," 2019 International Conference on Electrical Drives \& Power Electronics (EDPE), The High Tatras, Slovakia, 2019, pp. 318-322.
	%\bibitem{eeict2020} J. Miklas, ``Estimating the Power BJT Excess Charge Recombination Time Constant'', unpublished
	\bibitem{diplomovka} J. Miklas, {Power Switching Transistors}, Brno: Brno University of Technology, Faculty of Electrical Engineering and Communication. 2016. Head of Diploma Thesis doc. Dr. Ing. Miroslav Patocka
	\bibitem{prochazka} P. Prochazka, J. Miklas, I. Pazdera, M. Patocka, J. Knobloch, R. Cipin, ``Measurement of Power Transistors Dynamic Parameters'', Mechatronics 2017, pp.571-577, ISBN 978-3-319-65959-6


%\bibitem{b1} G. Eason, B. Noble, and I. N. Sneddon, ``On certain integrals of Lipschitz-Hankel type involving products of Bessel functions,'' Phil. Trans. Roy. Soc. London, vol. A247, pp. 529--551, April 1955.
%\bibitem{b2} J. Clerk Maxwell, A Treatise on Electricity and Magnetism, 3rd ed., vol. 2. Oxford: Clarendon, 1892, pp.68--73.
%\bibitem{b3} I. S. Jacobs and C. P. Bean, ``Fine particles, thin films and exchange anisotropy,'' in Magnetism, vol. III, G. T. Rado and H. Suhl, Eds. New York: Academic, 1963, pp. 271--350.
%\bibitem{b4} K. Elissa, ``Title of paper if known,'' unpublished.
%\bibitem{b5} R. Nicole, ``Title of paper with only first word capitalized,'' J. Name Stand. Abbrev., in press.
%\bibitem{b6} Y. Yorozu, M. Hirano, K. Oka, and Y. Tagawa, ``Electron spectroscopy studies on magneto-optical media and plastic substrate interface,'' IEEE Transl. J. Magn. Japan, vol. 2, pp. 740--741, August 1987 [Digests 9th Annual Conf. Magnetics Japan, p. 301, 1982].
%\bibitem{b7} M. Young, The Technical Writer's Handbook. Mill Valley, CA: University Science, 1989.
\end{thebibliography}


\end{document}
